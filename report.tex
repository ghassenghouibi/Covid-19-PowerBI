\documentclass[french,a4paper,11pt,oneside]{book}
\usepackage[utf8]{inputenc}
\usepackage{hyperref}
\usepackage{babel}
\usepackage{amsmath}
\usepackage{amsfonts}
\usepackage{amssymb}
\usepackage{graphicx}
\usepackage{xcolor}
\usepackage[rightlabels]{titletoc}
\usepackage{pdfpages}
\usepackage{grffile}
\usepackage[export]{adjustbox}
\usepackage{listings}
\usepackage{datetime} 
\usepackage{sectsty}
\renewcommand{\contentsname}{Sommaire}
\renewcommand{\thesection}{\Roman{section}} 

\titlecontents{subsection}% <section>
[2.35em]% <left>
{\small\contentsmargin{1.5em}}% <above-code>
{\thecontentslabel.\hspace{3pt}}% <numbered-entry-format>; you could also use {\thecontentslabel. } to show the numbers
{}% <numberless-entry-format>
{\enspace\titlerule*[0.5pc]{.}~\contentspage}% <filler-page-format>

\usepackage{float}
\usepackage{color,soul}

\definecolor{googleblue}{HTML}{0F9D58}
\definecolor{googlegreen}{HTML}{4285F4}
\definecolor{googlered}{HTML}{DB4437}
\definecolor{googleyellow}{HTML}{F4B400}

\sectionfont{\color{googlered} \normalfont} % sets colour of sections
\subsectionfont{\color{googlegreen} \normalfont} % sets colour of sections
\subsubsectionfont{\color{googleblue} \normalfont} % sets colour of sections

\author{\color{blue} \raggedright GHOUIBI GHASSEN}
\title{\color{red} \normalfont\huge Modélisation de Gestionnaire de stock}
\begin{document}
 \begin{titlepage}
  \centering 

  \vspace{1cm}

  \vspace{0.5cm}
  {\huge\bfseries Power BI\par}
  \vspace{0.5cm}
  \vfill
  {\Large\itshape \color{googlegreen} Data visualisation sur dataset de Covid-19\par}
  \vfill
  % Bottom of the page
  {\large Auteur \par
  GHOUIBI Ghassen\textsc{}\par}
  \vspace{1cm}
  {\large Encadré par \par
 Mme Rakia Jaziri\textsc{}\par
 
  }
 \end{titlepage}
	\tableofcontents
	
	\newpage
	\section{Introduction}{
		Ce rapport contient les étapes réaliser pour répondre au question du projet détaille en section, il sera accompagné du projet exporter.\\
		\subsection{Power BI}{
		Power BI\footnote{Cette défintion est prise de wikipedia \url{https://en.wikipedia.org/wiki/Power_BI}} est un service d'analyse commerciale de Microsoft. Il vise à fournir des visualisations interactives et des capacités de Business Intelligence avec une interface suffisamment simple pour que les utilisateurs finaux puissent créer leurs propres rapports et tableaux de bord.	
		}
		\subsection{Objectif}{
			Le choix de travailler sur le covid en respectant les consignes du rapport notre dataset est sous ce formet (20000,6).\\
			%TODO
			Notre objectif c'est prédire des résultats sur quelques jours à traves notre dataset.
		}
	
	}

	\section{Mise en place}{
		Aprés faire le choix sur l'outil qu'on va utiliser dans ce rapport, il est nécessaire quelques installation à faire avant d'installer Power BI.\\
			\subsection{Machine Virtuelle}{
				On fait les choix de configuration suivante aprés avoir télécharger \texttt {Windows 10} grâce au lien \href{https://www.microsoft.com/fr-fr/software-download/windows10ISO}{suivant}.
				
				\includegraphics[width=350pt]{config}
			}
			\subsection{Installation}{
				Aprés avoir installer notre \texttt{Windows 10} sur notre machine virtuelle et passer l'étape de configuration de windows on peut télécharger \texttt{Power Bi}.
				Soit on sur le \texttt{Microsoft store}, soit à traves ce \href{https://www.microsoft.com/fr-fr/download/details.aspx?id=58494}{lien}.
				Maintenant il faut juste ouvrir le logiciel est voici notre premier projet en Power BI.\\
				\includegraphics[width=350pt]{done}
			}
	}

	\section{Jeu de données}{
		
		\subsection{Chargement de données}{
		}
		\subsection{Nettoyage et transformation de données}{
		}
	
		\subsection{Dashbord intelligent}{
		
		}

		\subsection{Business Analytique}{
			proposer et traiter un cas d’usage analytique en mettant l’accent sur l’étape de visualisation,
		
		}
	
		\subsection{Business Intelligence}{
			Création de tableau de bord pour le suivi de l’activité de l’entreprise, Reporting, création de datamart, ....
		
		}
		

	}
	
	\section{Conclusion}{
	
	}


	
	

\end{document} 