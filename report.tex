\documentclass[french,a4paper,11pt,oneside]{book}
\usepackage[utf8]{inputenc}
\usepackage{hyperref}
\usepackage{babel}
\usepackage{amsmath}
\usepackage{amsfonts}
\usepackage{amssymb}
\usepackage{graphicx}
\usepackage{xcolor}
\usepackage[rightlabels]{titletoc}
\usepackage{pdfpages}
\usepackage{grffile}
\usepackage[export]{adjustbox}
\usepackage{listings}
\usepackage{datetime} 
\usepackage{sectsty}
\renewcommand{\contentsname}{Sommaire}
\renewcommand{\thesection}{\Roman{section}} 

\titlecontents{subsection}% <section>
[2.35em]% <left>
{\small\contentsmargin{1.5em}}% <above-code>
{\thecontentslabel.\hspace{3pt}}% <numbered-entry-format>; you could also use {\thecontentslabel. } to show the numbers
{}% <numberless-entry-format>
{\enspace\titlerule*[0.5pc]{.}~\contentspage}% <filler-page-format>

\usepackage{float}
\usepackage{color,soul}

\definecolor{googleblue}{HTML}{0F9D58}
\definecolor{googlegreen}{HTML}{4285F4}
\definecolor{googlered}{HTML}{DB4437}
\definecolor{googleyellow}{HTML}{F4B400}

\sectionfont{\color{googlered} \normalfont} % sets colour of sections
\subsectionfont{\color{googlegreen} \normalfont} % sets colour of sections
\subsubsectionfont{\color{googleblue} \normalfont} % sets colour of sections

\author{\color{blue} \raggedright GHOUIBI GHASSEN}
\title{\color{red} \normalfont\huge Modélisation de Gestionnaire de stock}
\begin{document}
 \begin{titlepage}
  \centering 

  \vspace{1cm}

  \vspace{0.5cm}
  {\huge\bfseries Power BI\par}
  \vspace{0.5cm}
  \vfill
  {\Large\itshape \color{googlegreen} Data visualisation sur l'activité d'une entreprise en événementielles\par}
  \vfill
  % Bottom of the page
  {\large Auteur \par
  GHOUIBI Ghassen\textsc{}\par}
  \vspace{1cm}
  {\large Encadré par \par
 Mme Rakia Jaziri\textsc{}\par
 
  }
 \end{titlepage}
	\tableofcontents
	
	\newpage
	\section{Introduction}{
		Ce rapport contient les étapes réaliser pour répondre au question du projet détaille en section, il sera accompagné du projet exporter.\\
		\subsection{Power BI}{
		Power BI\footnote{Cette défintion est prise de wikipedia \url{https://en.wikipedia.org/wiki/Power_BI}} est un service d'analyse commerciale de Microsoft. Il vise à fournir des visualisations interactives et des capacités de Business Intelligence avec une interface suffisamment simple pour que les utilisateurs finaux puissent créer leurs propres rapports et tableaux de bord.	
		}
		\subsection{Objectif}{
			On va traiter trois sujets avec PowerBi notre idée de base est la suivante visualiser les pertes suite au covid 19 dans une société d'évenementielle notre but tout d'abord et visualiser la courbe du covid-19 et entirer des informations concernant le temps ou les cas de décès (comme ci c'était un client perdu), aprés il faudra prévoir l'activité de l'entreprise et la répartition des clients et nombre commandes finalement visualiser les événements récents pour prédire une reprise de l'activité de l'entreprise et mettre une stratégie en place.
		}
	
	}
	 \newpage
	\section{Mise en place}{
		Aprés faire le choix sur l'outil qu'on va utiliser dans ce rapport, il est nécessaire quelques installation à faire avant d'installer Power BI.\\
			\subsection{Machine Virtuelle}{
				On fait les choix de configuration suivante aprés avoir télécharger \texttt {Windows 10} grâce au lien \href{https://www.microsoft.com/fr-fr/software-download/windows10ISO}{suivant}.
				
				\includegraphics[width=350pt]{config}
			}
			\subsection{Installation}{
				Aprés avoir installer notre \texttt{Windows 10} sur notre machine virtuelle et passer l'étape de configuration de windows on peut télécharger \texttt{Power Bi}.
				Soit on sur le \texttt{Microsoft store}, soit à traves ce \href{https://www.microsoft.com/fr-fr/download/details.aspx?id=58494}{lien}.
				Maintenant il faut juste ouvrir le logiciel sur le desktop.
			}
	}
	\newpage

	\section{Jeu de données}{
		
		\subsection{Chargement de données}{
			\subsubsection{Covid-19}{
				Aprés avoir lancer \texttt{Power BI} on va cliquer sur {\itshape Obtenir les données} ensuite mettre le choix sur excel vu que nos données sont du format xlsx, aprés cliquer sur tranformer les données et voici un exemple :\\
				\includegraphics[width=350pt]{data}
				\\
				Nos données sont importer dans \texttt{Power Query},On effectue un anaylse rapide sur notre dataset.
				On s'intresser aux 5 premiers collones pour visualiser les résultats, on remarque que notre dataset est propre et ne nécissite pas d'actions de notre part.\\
				
			}
			\subsubsection{Entreprise}{
				De la même manière on va importer nos données d'entreprise sauf que ici le choix s'est porte sur une entreprise de livraisons malheureusement on n'a pas réussi a trouver un dataset qui nous permet de réaliser notre objectif.
				En effet on va considerer que les commandes effectuer ici sont des commandes de billets pour un événement.\\
				\includegraphics[width=350pt]{data2}
			}
		
		}
		\subsection{Nettoyage et transformation de données}{
 			Comment préciser dans la section d'avant, en effet notre dataset est propre est prêt sauf qu'il y a quelque valeurs qui sont faussés à cause des accents français.\\
 			On clique sur remplacer les valeurs et remplacer ses valeurs la.\\
 			\includegraphics[width=350pt]{replace}
 			
		}
		\newpage
		\subsection{Dashbord intelligent}{
			
			\subsubsection{Covid-19}{
				Dans cette section on souhaite réaliser un dashbord intelligent comme celui de google.\\
				\includegraphics[width=450pt]{dashbord}
				On a commencer par trouver les nombres infections par jours au cours du temps, ensuite pareil pour le nombre de morts en utilisant l'onglet questions réponses.
				Pour visualiser la répartition dans le monde il faut mettre le choix sur la carte et aussi sur quel base on a choisi les infections.\\
				Pour determiner le maximum de morts par pays on a choisi les graphiques à barres.\\
				Finalement pour regrouper les cas de décès d'infections et de guérisons, il suffit de cliquer sur un pays pour visualiser tous ces courbes, Voici le résultat obtenu:\\
				\includegraphics[width=450pt]{dash1}
				\\
				Voici un exemple sur la France: \\
				\includegraphics[width=450pt]{dash2}
			
			}
			\subsubsection{Business Analytique/BI}{
				Comment détaillé au par avant on va se concentrer sur notre deuxième dataset qui comporte les commandes effectués.\\
				\includegraphics[width=450pt]{dash3}
				Dans cette étape on a visualiser les profits par an on constate déjà c'est une entreprise en plein croissance.\\
				D'ou que les graphiques à barres groupes montre l'évolution du profit au cours de l'année sur l'ensemble de ventre.\\
				On réalise aussi que la majorité de commande sont livré en class standard,et l'état de California qui constitue une grande part du marché devant l'état de New York.	
			}
		
			\subsubsection{événement}{
				On va supprimer beaucoup de champs qui ne sont pas utiles pour notre cas, on s'est focaliser sur la visualisation malheureusement il y a avait un problème pour afficher les coordonnées du GPS, on a visualiser le nombre d'événement par mois et séléctionner les lieux les plus fréquent, et la catégorie de l'évenement finalement on a optér pour le prix voici le dashboard correspondant:\\
				\includegraphics[width=450pt]{dash4}
			
			}
			
			
						
		}

	
		

	}
	\section{Conclusion}{
		En comparaison avec DATAIKU que je trouve à mon avis largement supérieur niveau rapidité et fonctionnalités plus avancer néan moins Power Bi et trés orienté BI du point de vu d'un développeur.\\
		Finalement Power BI présenté beaucoup de fonctionnalités à connaître pour le futur qui est un outils de premier rang pour une jeune entreprise.
	}


	
	

\end{document} 